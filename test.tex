\documentclass{iscram}
\iscramset{
	title={Automatic user generated image collecting as database for NEOHAZ 3D flood level mapping
project},
	short title={Final Assignment Disaster Mapping},
	author={
		short name={J. Krauth},
		full name={Julian Krauth},
		affiliation={Heidelberg University
		\\Study course: Geography
		\\j.krauth@uni-heidelberg.de},
	},
}
\addbibresource{example.bib}
\begin{document}
\maketitle
\abstract{Based as an practical approach to automatically collect potential content in order to contribute
to the NEOHAZ 3D flood level mapping project. As input source, several social-media
platforms could be analyzed in terms of their content suitability and availability.
The main idea is to create a python script which filters social-media messages spatially (to
area of interest) and semantically (specific image tags, here e.g. flood, floodplain).
Furthermore, with this approach, different time aspects for flood level mapping could be
analyzed (if image quantity is sufficient).
A main source of suitable images may be the social-network platform FlickR, with it´s
detailed image metadata (precise timestamps, sensor information) as well as additional
information like autogenerated tags which could be used to identify buildings and at last the
easy-to- operate python api.
Aim is to create a framework that is capable to collect images from different areas/times with
minor changes to the script, for example a shape-file with a bounding box of another area of
interest.}
\keywords{Some keywords}
\section{First section}
\subsection{First subsection}
\subsubsection{First subsubsection}
Bla bla \parencite{key} \ldots{}
\printbibliography
\end{document}